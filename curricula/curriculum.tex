\documentclass[notitle,nologo]{europecv}
\usepackage[a4paper,top=1.27cm,left=1cm,right=1cm,bottom=2cm]{geometry}
\usepackage[english]{babel}
\usepackage{url}
\usepackage{relsize}

\usepackage{tikz}

\usepackage[colorlinks=false, pdfborder={0 0 0}]{hyperref}

\hypersetup{
    pdfinfo={
        Title={Curriculum Vitae},
        Subject={ },
        Author={Paolo Bolzoni},
        Creator={ },
        Producer={ },
    }
}

\def\ifmonospace{\ifdim\fontdimen3\font=0pt }
\def\Cpp{%
    \ifmonospace%
    C++%
    \else%
    C\kern-.1em\raise.40ex\hbox{\smaller[2]{++}}%
    \fi%
\spacefactor1000 }
\def\PhD{{\smaller[0.5]P}h{\smaller[0.5]D}}
\def\POI{{\smaller[0.5]POI}}
\newcommand\Acr[1]{{\smaller[0.5]#1}}


\usepackage{fontspec}
\defaultfontfeatures{Ligatures=TeX, Scale=MatchLowercase }
\setmainfont{Tinos}
\setsansfont{Arimo}
\setmonofont{Source Code Pro}
\newfontface\jfont{IPAexMincho}
\selectfont


\ecvlastname{BOLZONI}
\ecvfirstname{Paolo Ph.D.}
\ecvaddress{Via Provinciale 4; 43018 Sissa Trecasali PR; Italy}
\ecvtelephone{+39 3470683057}
\ecvfax{}
\ecvemail{paolo.bolzoni.brown@gmail.com}
\ecvnationality{Italian}
\ecvdateofbirth{1981--06--02}
\ecvgender{Male}
\ecvbeforepicture{\raggedleft}
\ecvpicture[height=45mm]{picture.jpg}
\ecvafterpicture{\ecvspace{-45mm}}

\begin{document}
\selectlanguage{english}

\begin{europecv}

\ecvpersonalinfo %[4mm]

\ecvsection{Work experience}
\ecvitem{2021/02}{High-School Computer Science Teacher, Viadana, Mantova}
\ecvitem{2019/12--2020/12}{Software developer and R\&D for Cagla Inc., Toyota City}
\ecvitem{2018/09--2019}{Postdoctoral Researcher at Dublin City University}
\ecvitem{2018/03--2018/09}{Software developer and R\&D for Energee3, Reggio Emilia}
\ecvitem{2018/01}{Preparing Lesson Material for teaching \Acr{MISRA}~C 2012 guidelines}
\ecvitem{2016--2017}{Teaching Assistant for Information Security course (twice) at Uni\Acr{BZ}}
\ecvitem{2013--2017}{\PhD{} Student, Free University of Bolzano--Bozen}
\ecvitem{2009--2012}{Tutoring (on voluntary basis) of Uni\Acr{PR} students for \Cpp{} and Algorithms and Data Structures}
\ecvitem{2005--2012}{Week-end waiter, Agriturismo al Cason}
\ecvitem{2004--2008}{Summer jobs}
\ecvitem{2002--2004}{Network Manager, Manghi S.p.A.}


\ecvsection{Education}
\ecvitem{2018/05}{Computer Science Ph.D.}
\ecvitem{2015--2016}{Study exchange with the Nagoya University (\jfont{名古屋大学})}
\ecvitem{2014}{Participated to the 17th \Acr{IPCO} summer school and conference (Integer Programming and Combinatorial Optimization)}
\ecvitem{2013--2017}{\PhD{} Student, Free University of Bolzano--Bozen}
\ecvitem{Research topic}{Itinerary planning for Tourist Applications}
\ecvitem{2012}{Master Degree in Computer Science 110/110 cum laude, Universit\`a degli Studi di Parma}
\ecvitem{2009}{Bachelor Degree in Computer Science 103/110, Universit\`a degli Studi di Parma}
\ecvitem{2002}{Completed a post high-school accountancy course}
\ecvitem{2000}{Completed high-school in accountancy}


\ecvsection{Publications}
\ecvitem{\Acr{FGCS SI} 2019}{Improving Orienteering-based Tourist Trip Planning with Social Sensing}
\ecvitem{\Acr{SSTD} 2017}{Hybrid Best-First Greedy Search for Orienteering with Category Constraints}
\ecvitem{\Acr{DEXA} 2017}{Itinerary Planning with Category Constraints Using a Probabilistic Approach}
\ecvitem{\Acr{GIS}cience 2016}{Fast Computation of Continental-Sized Isochrones}
\ecvitem{\Acr{ACM SIGSPATIAL} 2014}{Efficient Itinerary Planning with Category Constraints}


\ecvsection{Technical skills and competences}
\ecvitem{Systems}{\Acr{GNU}/Linux, Docker, Postgis}
\ecvitem{Development}{\Cpp, Bash, Python, Java, SQL, \LaTeX, front and backend Javascript}
\ecvitem{Techniques}{Object Oriented, functional, generic patterns, idiomatic code, command line}
%\ecvitem{Driving license}{Italian B (cars)}

\ecvsection{Talks}
\ecvitem{2017}{Prepared the talks for \emph{Hybrid Best-First Greedy Search for
Orienteering with Category Constraints} paper at \Acr{SSTD} 2017 and
for \emph{Itinerary Planning with Category Constraints Using a Probabilistic
Approach} paper at \Acr{DEXA} 2017}
\ecvitem{2014}{Presented \emph{Efficient Itinerary Planning with Category
Constraints} paper at \Acr{ACM SIGSPATIAL} conference}
\ecvitem{2012}{Presented my \PhD{} work at the DB retreat in Racines, which was a
joint event of the Universities of Zurich and Bolzano--Bozen}
\ecvitem{2009}{Code good practices and code cleanness guidelines presentation
in the context of \Acr{ECLAIR}\footnote{\url{https://en.wikipedia.org/wiki/ECLAIR}}, a platform for
the static analysis of C programs}


\ecvsection{Research and thesis work}
\ecvitem{\PhD{} research topic}{\emph{Itinerary planning for Tourist Applications}}
\ecvitem{}{Premade touristic tours are often sub-optimal because the taste of
the tourists are very varied. To implement a tourist tour guide we are studying
and developing new approximate algorithms for the Orienteering with Categories
problem. It is an extension of the \Acr{NP}-hard Orienteering problem that
in its turn is a mix between the Traveling Salesman and the Knapsack problems.
The first approach to solve this problem using clusters had brought to the
publication of the paper \emph{Efficient Itinerary Planning with Category
Constraints} at \Acr{ACMGIS GIS} Dallas conference in 2014.  The second
approach uses a priority deque on a custom solution space and had brought to
the paper \emph{Hybrid Best-First Greedy Search for Orienteering with Category
Constraints} at \Acr{SSTD} 2017.  An experimental approach using
probabilistic graph pruning had brought to the paper \emph{Itinerary Planning
with Category Constraints Using a Probabilistic Approach} at \Acr{DEXA}
2017. The idea of extending the concept of memoization from functions results
to algorithms steps had brought to a tangentially related result of speeding up
isochrone computations and the paper \emph{Fast Computation of
Continental-Sized Isochrones} at GIScience 2016.}

\ecvitem{Master Thesis}{\emph{Design and implementation of a distributed system
for computations on matrices with entries in semirings}}
\ecvitem{}{The aim of a distributed system for semiring-matrices computation is
to implement classic graph algorithms using mathematical computations directly
over adjacency matrices as they are easier to parallelize than dynamic
programming approaches.  The work was about design and implement a
message-based implementation using the \Acr{\O MQ} concurrency framework.}

\ecvitem{Bachelor Thesis}{\emph{Automatic Checking of Coding Rules}}
\ecvitem{}{Analysis of good coding rules for the C and \Cpp{} languages from
various sources, including \Acr{MISRA} and \Acr{HICPP} in order to
understand the algorithmic needs to make an automatic checker and the relation
between those checkers and rules violations. For example, to make an automatic
checker for a rule like ``Do not use \texttt{goto}'' it only needs the token
stream; instead for a rule like ``The \texttt{U} suffix shall used for all
unsigned integers literals'' it needs information from the hardware and the C
type system.}

\ecvsection{Languages}
%\ecvmothertongue{Italian}

%\ecvlanguage{English}{\ecvBTwo}{\ecvCOne}{\ecvBTwo}{\ecvCTwo}{\ecvCTwo}
\ecvitem{\textbf{Italian}}{Native}
\ecvitem{\textbf{English}}{Fluent}
\ecvitem{\textbf{Japanese}}{Beginner}

\ecvitem{}{}

\ecvitem{}{{\bf Personal Interest}}
\ecvitem{}{Meditation, hiking, programming, travels, board games, reading}

\end{europecv}

\end{document}
